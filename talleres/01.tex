\usemodule[pre-01]

\mainlanguage[es]
\usemodule[chart]
\setupcolors[state=start]
\definelayer[LogoUTM]
%\setupbodyfont[cmr,12pt]
\setuptype[color=blue]
\setuphead[subject][
		style=\tfa\bf
		]
%\setuphead[subsubject][
	%	style=\bf
	%	]
\setuplayout[
		backspace=1cm,
		width=16cm
		]
		
\starttext

\StartTitlePage
  \CONTEXT ~~\&~~TeXworks\\
  
{\tfxx
\goto{http://wiki.contextgarden.net}[url(http://wiki.contextgarden.net)]}

\blank

Taller 01
\blank[]
Instalación y Comandos Básicos
\blank[3cm]      

\CONTEXT -ec\par
      
{\externalfigure[figuras/utm_logo][width=3cm]}
\StopTitlePage 

\Topics{Contenido}

\Topic{Instalación y configuración}


% INSTALAR CONTEXT

\subject{Primer paso:  instalar \CONTEXT}\\


\startitemize[n, packed]

\item Descargar el archivo ZIP del siguiente enlace\par  \goto{http://wiki.contextgarden.net}[url(http://wiki.contextgarden.net)].

\item Guardar este archivo en el Disco local C: y extraer en este directorio. Aparecerá una carpeta de nombre {\ss{context}}, en la cual se hallará un archivo de nombre {\ss{first-setup.bat}}
\item Ejecutar como administrador el archivo {\ss{first-setup}}. Se abrirá una terminal de Windows que mostrará los archivos descargados. 

\item Esperar a que la ventana se cierre automáticamente. Este proceso puede tardar según la velocidad de conexión a internet.

\stopitemize


%INSTALAR TEXWORKS

\subject{Segundo paso: instalar TeXworks}\\


\startitemize[n, packed]
\item Descargar el archivo ejecutable del siguiente enlace:\par  \goto{http://www.tug.org/texworks/}[url(http://www.tug.org/texworks/)]

\item Ejecutar el archivo e instalar con los valores por defecto.

\item Ya instalado, ejecutar el programa {\ss{TeXworks}}, dirigirse al menú  {\ss{Editar}} y a la opción {\ss{Preferencias}}.

\item Una vez abierta la ventana de {\ss{Preferencias}}, dirigirse a la pestaña {\ss{Compilación}} y en esta a la sección {\ss{Ubicación de TeX y programas relativos}}, en la que se encuentra un botón con el signo {\ss\bf{+}}, sobre el cual hay que hacer click.

\item Se abrirá una nueva ventana, en esta se realizará la ruta a continuación:\\ {\ss{Disco local C: ---> context ---> tex ---> texmf-mswin ---> bin}}.\\ Luego, dar click en el botón {\ss{seleccionar carpeta}}.

\item Una vez que aparezca el directorio {\ss{C:/context/tex/texmf-mswin/bin}} en la lista, llevarlo hasta la {\bf{primera}} posición utilizando el botón con la flecha verde hacia arriba.

\item En la misma ventana de {\ss{Preferencias}}, en la sección {\ss{Herramientas para procesamiento}} seleccionar como predeterminada la opción {\ss{ConTeXt (LuaTeX)}}. El resultado hasta este punto debe coincidir con lo mostrado en la Figura 1 (\in{Fig.}[fig:f1]). Dar click en el botón {\ss{OK}}

\stopitemize


\placefigure
[]
[fig:f1]
{Correcta configuración de las preferencias de TeXworks}
{\externalfigure[figuras/setup][width=.67\textwidth]}

%ATAJOS DEL TECLADO

\subject{Tercer paso: atajos del teclado}

\startalignment
\startitemize[n,packed]

\item Entrar a la siguiente dirección web: \goto{https://github.com/farliz/texworks-context/blob/master/tw-context.txt}[url(https://github.com/farliz/texworks-context/blob/master/tw-context.txt)]. 

\item Crear un archivo de texto en el escritorio con el nombre de {\ss{tw-context}} y pegar el código expuesto en la página web en este archivo.

\item Dirigirse al menú {\ss{Scripts}} de TeXworks, luego a la opción {\ss{Scripts de TeXworks}} y luego a la opción {\ss{Carpeta de Scripts}}. Se abrirá una ventana, dentro de la cual hay que regresar un directorio, el cual contendrá la carpeta {\ss{completions}} donde yace un archivo de texto de nombre {\ss{tw-context}}.

\item Borrar este archivo de la carpeta y en su lugar pegar el archivo que creamos en el escritorio.

\stopitemize
\stopalignment

%COMANDOS BÁSICOS
\Topic{Comandos básicos}

%SECCIÓN DE TEXTO
\type{\starttext}\\
Hola mundo, este es un texto de prueba\\
\type{\stoptext}

%FORMA DE UTILIZAR LOS COMANDOS
\subject{Comandos}
 Los comandos  empiezan con una barra inclinada hacia la izquierda\type{\}, seguido del
   nombre del comando. Entre corchetes \type{[ ]} se colocan las opciones del comando y
   entre llaves \type{{ }} los argumentos. El símbolo \% indica que es n comentario o
   anotación y no lo reconoce como comando.
\blank
   Ejemplo:\par

   \type{\chapter{Introducción}}\par
   \type{\color[red]{texto}}\par
   \type{{\bf texto}}\page       

\subject{Comandos \quote{start-stop}}   

Otra forma de ejecutar comandos en la forma \type{start-stop}.

\type{\startitemize[n, packed]}\\
\type{\item}  Uno\\
\type{\item}  Dos\\
\type{\item}  Tres\\
\type{\stopitemize}

%TABLA DE CONTENIDOS
\subject{Insertar tabla de contenidos}
\type{\placecontent}\par
\type{\completecontent}
\page

%ESTRUCTURA DEL DOCUMENTO SIN NUMERACIÓN
\subject{Estructura del documento sin numeración}
\type{\title{Título}}: Título\\
\type{\subject{Tema}}: Tema\\
\type{\subsubject{Subtema}}: Subtema\\
\type{\subsubsubject{Subsubtema}}: Subsubtema\\

%ESTRUCTURA DEL DOCUMENTO CON NUMERACIÓN
\subject{Estructura del documento con numeración}
\type{\chapter{Capítulo}}: 1 Capítulo\\
\type{\section{Sección}}: 1.1 Sección\\
\type{\subsection{Subsección}}: 1.1.1 Subsección\\
\type{\subsubsection{Subsubsección}}: 1.1.1.1 Subsubsección\\

\subject{Comandos útiles}
% COMANDOS UTILES
\type{\blank } --> Espacio en blanco\\

\type{\par} --> Salto de línea\\

\type{\\} --> Salto de línea\\

\type{\page} --> Salto de página

\subject{Caracteres especiales}

Algunos caracteres se deben colocar anteponiendo la barra inclinada:

\type{\%} --> \%



%FAMILIAS DE LETRAS
\subject{Familias de letras}
~~~~~~~~~~~~~~~~~~~~~~~~Texto normal\\
\type{{\rm texto}}: {\rm{Romanas}\\
\type{{\ss texto}}: {\ss{Sans Serif}}\\
\type{{\sc texto}}: {\sc{Versalitas}}\\
       
%ESTILO DE LETRAS
\subject{Estilo de letras}
~~~~~~~~~~~~~~~~~~~~~~~~Texto normal\\
\type{{\bf texto}}:   {\bf{Negritas}}\\
\type{{\it texto}}:    {\it{Itálicas}}\\
\type{{\it texto}}: {\em{Énfasis}}\\
\type{{\sl texto}}:  {\sl{Inclinadas}}\\
\type{{\overbar texto}}: {\overbar{Sobrerayado}}\\
\type{{\underbar texto}}: {\underbar{Subrayado}}\\
\type{{\overstrike texto}}: {\overstrike{Tachado simple}}
\page

%COMILLAS
\subject{Comillas simples}
\type{\quote{simple}} ---> \quote{simple}
\subject{Comillas inglesas}
\type{\startquotation}\\
El diablo se esconde en los detalles}\\
\type{\stopquotation}
\blank 
Da como resultado:
\startquotation El diablo se esconde en los detalles \stopquotation
\blank
\subject{Comillas latinas}
Atl + 174 ---> «\\
Alt + 175 ---> »\\
\blank
\type{\setuplanguage}[es]\\
[leftquotation=«, rightquotation=», leftquote=', rightquote=']
\page

%TAMAÑO DE LETRA
\subject{Tamaño de letra}
~~~~~~~~~~~~~~~~~~~~~~~~Texto por defecto\\
\type{\tfxx{texto}}:  {\tfxx{texto}}\\
\type{\tfx{texto}}:   {\tfx{texto}}\\
\type{\tfa{texto}}:    {\tfa{texto}}\\
\type{\tfb{texto}}:   {\tfb{texto}}\\
\type{\tfc{texto}}:  {\tfc{texto}}\\
\type{\tfd{texto}}:  {\tfd{texto}}\\

%LISTAS
\subject{Listas}
\type{\startitemize[n, packed] %[n, R, r, a , A, 1, 2, 3]}\\
\type{\item}  Uno\\
\type{\item}  Dos\\
\type{\item}  Tres\\
\type{\stopitemize}


\bTABLE[frame=off]

\bTR
\bTD[width=3cm] 
\startitemize[n, packed]
\item Uno
\item Dos
\item Tres
\stopitemize
\eTD
\bTD[width=3cm] 
\startitemize[R, packed]
\item Uno
\item Dos
\item Tres
\stopitemize
\eTD
\bTD [width=3cm]
\startitemize[a, packed]
\item Uno
\item Dos
\item Tres
\stopitemize
\eTD
\bTD [width=3cm]
\startitemize[3, packed]
\item Uno
\item Dos
\item Tres
\stopitemize
\eTD

\eTR
\eTABLE

%ELEMENTOS DEL TEXTO
\subject{Elementos del texto}
Superíndice ---> m\type{\high{2}} ---> m\high{2}\blank
Subíndice ---> CO\type{\low{2}} ---> CO\low{2}\blank
Super y subíndice combinados ---> NH\type{\lohi{4}{+}} ---> NH\lohi{4}{+}

%NOTAS AL PIE
\subject{Notas al pie}
Texto\type{\footnote{Contenido de la nota al pie}} ---> Texto\footnote{Contenido de la nota al pie}.

% ECUACIONES
\subject{Matemáticas}

Las ecuaciones se insertan  entre dos símbolos de dolar \$ \$.

\starttyping
$ c^2 = a^2 + b^2 $
\stoptyping

{\bf Resulta:}\\
$ c^2 = a^2 + b^2 $


%ALINEACIÓN DEL TEXTO
\subject{Alineación del texto}
\subsubject{Alineación de una sola linea}
\rightaligned{\type{\rightaligned{Texto}}}\\
\rightaligned{Texto}\\

\middlealigned{\type{\middlealigned{Texto}}}\\
\middlealigned{Texto}\\

\leftaligned{\type{\leftaligned{Texto}}}\\
\leftaligned{Texto}\\

\subsubject{Alinear uno o varios párrafos}

\startalignment[left]
\type{\startalignment[left]}\\
La sección entre estos comandos pasa a estar alineado según se indique (middle,left,right o vacío para justificar)\\
\type{\stopalignment}\\
\stopalignment


%TEXTO DELINEADO O SOMBREADO
\subject{Texto delineado o sombreado}
\type{\framed{Texto Texto Texto}} ---> \framed{Texto Texto Texto}
\blank

\type{\startframed}\\
La ley de la gravedad no es responsable de que la gente se enamore
\type{\stopframed}
\blank
Resulta:\\
\startframed
La ley de la gravedad no es responsable de que la gente se enamore
\stopframed
\blank
\type{\startframed[background=color, backgroundcolor=gray]}\\
La ley de la gravedad no es responsable de que la gente se enamore
\type{\stopframed}
\blank
Resulta:\\
\startframed[background=color, backgroundcolor=gray]
La ley de la gravedad no es responsable de que la gente se enamore
\stopframed
\page

\type{\startframed[background=color, backgroundcolor=black, foregroundcolor=white]}\\
La ley de la gravedad no es responsable de que la gente se enamore
\type{\stopframed}
\blank
Resulta:\\
\startframed[background=color, backgroundcolor=blue, foregroundcolor=white]
La ley de la gravedad no es responsable de que la gente se enamore
\stopframed
\blank


\type{\startframed[background=color, backgroundcolor=black, foregroundcolor=blue,
    frame=off, corner=round]}\\
La ley de la gravedad no es responsable de que la gente se enamore
\type{\stopframed}
\blank
Resulta:\\
\startframed[background=color, backgroundcolor=cyan, foregroundcolor=blue, frame=off, corner=round]
La ley de la gravedad no es responsable de que la gente se enamore
\stopframed
\page

%COLOR DE LAS LETRAS
\subject{Color de las letras}
\type{\color[red]{Texto en rojo}}\\ \color[red]{Texto en rojo}\blank
\type{\color[green]{Texto en verde}}\\ \color[green]{Texto en verde}\blank
\type{\color[darkmagenta]{Texto en magenta oscuro}}\\ \color[darkmagenta]{Texto en magenta oscuro}
\blank[2cm]

Mas colores en \goto{http://wiki.contextgarden.net/Color}[url(http://wiki.contextgarden.net/Color)]
\page

\Topic{Configuración}%%%%%%%%%%%%%%%%%%%%%%%%%%

Los comandos de configuración empiezan con la palabra \type{\setup...}, y se colocan antes de \type{\starttext}.

% TIPO y TAMAÑO DE LETRA
\subject{Tipo y Tamaño de letra}
\type{\setupbodyfont[schola, 11pt]}\\


Otros tipos de letra:\par
\startitemize[packed, columns, three]
\item heros (sans serif)
\item termes
\item pagella
\item iwona
\item adventor
\item bonum
\item cmr (por defecto)
\item chorus
\item getium
\item cambria
\item utopia
\item antykwa
\item lbr
\stopitemize
\page

%ENCABEZADO Y PIE DE PÁGINA
\subject{Encabezado}
\type{\setupheader[state=start,style=\bf]}\\
\type{\setupheadertexts[Universidad X][Página \pagenumber]} 

\subject{Pie de página}
\type{\setupfooter[state=start,style=\bf]}\\
\type{\setupfootertexts[Universidad X][Página \pagenumber]}\blank[2*line]

Cualquiera de estas configuraciones resultará en un encabezado o pie de página como el siguiente:\blank[big]
\hairline
{\bf Universidad X ~~~~~~~~~~~~~~~~~~~~~~~~~~~~~~~~~~~~
~~~~~~~~~~~~~~~~~~~~~~~~~~~~~~~~~~~~~~~~~~~~~~~Página 2}
\hairline
\page

%INTERLINEADO
\subject{Interlineado}
\type{\setupinterlinespace[small]  % medium, big}
\blank[big]
\setupinterlinespace[small]
\input khatt-en\blank
\setupinterlinespace[big]
\input khatt-en
\page

%SANGRÍA
\setupinterlinespace[small]
\subject{Sangría}
\type{\setupindenting[yes, small]   % medium, big }
\blank[big]
Primer párrafo\blank

\setupindenting[yes,small]
\input khatt-en\blank
\setupindenting[yes,big]
\input khatt-en
\page

\setupindenting[no]
%COLUMNAS
\subject{Columnas}
%Comando en el preámbulo (antes de \type{\starttext}):\\
%\type{\setupcolumns[n=2]}%, align=no, distance=0.4cm, rule=on]}
\blank
%Comando en la sección de texto:\\
\type{\startcolumns}[n=2]\\
   \type{\input} davis\\
\type{\stopcolumns}\\
\blank
{\bf Resulta:}\\
\startcolumns[n=2]
\input davis
\stopcolumns
\page

%HIPERVÍNCULOS
\subject{Hipervínculos}
\type{\setupinteraction[state=start]}\\
\type{\goto{Texto del hipervínculo}[url(Enlace web)]}\blank[big]

\subsubject{Ejemplo}

...más información en:\\
\type{\goto{ConTeXt Garden}}\\
\type{[url(http://wiki.contextgarden.net/Main_Page)]}
\blank

Da como resultado: \blank
...más información en:\\ 
\goto{ConTeXt Garden}[url(http://wiki.contextgarden.net/Main_Page)]

\stoptext
